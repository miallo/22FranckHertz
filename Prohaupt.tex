% Für Bindekorrektur als optionales Argument "BCORfaktormitmaßeinheit", dann
% sieht auch Option "twoside" vernünftig aus
% Näheres zu "scrartcl" bzw. "scrreprt" und "scrbook" siehe KOMA-Skript Doku
\documentclass[12pt,a4paper,titlepage,headinclude,bibtotoc]{scrartcl}


%---- Allgemeine Layout Einstellungen ------------------------------------------

% Für Kopf und Fußzeilen, siehe auch KOMA-Skript Doku
\usepackage[komastyle]{scrpage2}
\pagestyle{scrheadings}
\setheadsepline{0.5pt}[\color{black}]
\automark[section]{chapter}


%Einstellungen für Figuren- und Tabellenbeschriftungen
\setkomafont{captionlabel}{\sffamily\bfseries}
\setcapindent{0em}


%---- Weitere Pakete -----------------------------------------------------------
% Die Pakete sind alle in der TeX Live Distribution enthalten. Wichtige Adressen
% www.ctan.org, www.dante.de

% Sprachunterstützung
\usepackage[ngerman]{babel}

% Benutzung von Umlauten direkt im Text
% entweder "latin1" oder "utf8"
\usepackage[utf8]{inputenc}

% Pakete mit Mathesymbolen und zur Beseitigung von Schwächen der Mathe-Umgebung
\usepackage{latexsym,exscale,stmaryrd,amssymb,amsmath}

% Weitere Symbole
\usepackage[nointegrals]{wasysym}
\usepackage{eurosym}

% Anderes Literaturverzeichnisformat
%\usepackage[square,sort&compress]{natbib}

% Für Farbe
\usepackage{color}

% Zur Graphikausgabe
%Beipiel: \includegraphics[width=\textwidth]{grafik.png}
\usepackage{graphicx}

% Text umfließt Graphiken und Tabellen
% Beispiel:
% \begin{wrapfigure}[Zeilenanzahl]{"l" oder "r"}{breite}
%   \centering
%   \includegraphics[width=...]{grafik}
%   \caption{Beschriftung} 
%   \label{fig:grafik}
% \end{wrapfigure}
\usepackage{wrapfig}

% Mehrere Abbildungen nebeneinander
% Beispiel:
% \begin{figure}[htb]
%   \centering
%   \subfigure[Beschriftung 1\label{fig:label1}]
%   {\includegraphics[width=0.49\textwidth]{grafik1}}
%   \hfill
%   \subfigure[Beschriftung 2\label{fig:label2}]
%   {\includegraphics[width=0.49\textwidth]{grafik2}}
%   \caption{Beschriftung allgemein}
%   \label{fig:label-gesamt}
% \end{figure}
\usepackage{subfigure}

% Caption neben Abbildung
% Beispiel:
% \sidecaptionvpos{figure}{"c" oder "t" oder "b"}
% \begin{SCfigure}[rel. Breite (normalerweise = 1)][hbt]
%   \centering
%   \includegraphics[width=0.5\textwidth]{grafik.png}
%   \caption{Beschreibung}
%   \label{fig:}
% \end{SCfigure}
\usepackage{sidecap}

% Befehl für "Entspricht"-Zeichen
\newcommand{\corresponds}{\ensuremath{\mathrel{\widehat{=}}}}
% Befehl für Errorfunction
\newcommand{\erf}[1]{\text{ erf}\ensuremath{\left( #1 \right)}}

%Fußnoten zwingend auf diese Seite setzen
\interfootnotelinepenalty=1000

%Für chemische Formeln (von www.dante.de)
%% Anpassung an LaTeX(2e) von Bernd Raichle
\makeatletter
\DeclareRobustCommand{\chemical}[1]{%
  {\(\m@th
   \edef\resetfontdimens{\noexpand\)%
       \fontdimen16\textfont2=\the\fontdimen16\textfont2
       \fontdimen17\textfont2=\the\fontdimen17\textfont2\relax}%
   \fontdimen16\textfont2=2.7pt \fontdimen17\textfont2=2.7pt
   \mathrm{#1}%
   \resetfontdimens}}
\makeatother

%Honecker-Kasten mit $$\shadowbox{$xxxx$}$$
\usepackage{fancybox}

%SI-Package
\usepackage{siunitx}

%keine Einrückung, wenn Latex doppelte Leerzeile
\parindent0pt

%Bibliography \bibliography{literatur} und \cite{gerthsen}
%\usepackage{cite}
\usepackage{babelbib}
\selectbiblanguage{ngerman}

\begin{document}

\begin{titlepage}
\centering
\textsc{\Large Anfängerpraktikum der Fakultät für
  Physik,\\[1.5ex] Universität Göttingen}

\vspace*{3.2cm}

\rule{\textwidth}{1pt}\\[0.5cm]
{\huge \bfseries
  Versuch Nr. 22 Franck-Hertz-Versuch\\[1.5ex]
  Protokoll}\\[0.5cm]
\rule{\textwidth}{1pt}

\vspace*{2.5cm}

\begin{Large}
\begin{tabular}{ll}
Praktikant: &  Michael Lohmann\\
 &  Felix Kurtz\\
% &  Kevin Lüdemann\\
 E-Mail: & m.lohmann@stud.uni-goettingen.de\\
 &  felix.kurtz@stud.uni-goettingen.de\\
% &  kevin.luedemann@stud.uni-goettingen.de\\
 Betreuer: & \\
 Versuchsdatum: & 10.03.2015\\
\end{tabular}
\end{Large}

\vspace*{0.8cm}

\begin{Large}
\fbox{
  \begin{minipage}[t][2.5cm][t]{6cm} 
    Testat:
  \end{minipage}
}
\end{Large}

\end{titlepage}

\tableofcontents

\newpage

\section{Einleitung}
\label{sec:einleitung}
Das \textsc{Bohr}sche Atommodell war das erste, welches eine quantenmechanische Betrachtung des Atomes vornahm.
Viele Zeitgenossen sahen es daher eher kritisch.
Um so wichtiger, dass mit dem Franck-Hertz-Versuch 1913 erstmals eine experimentelle Überprüfung erfolgte.
\cite{lp22}

\section{Theorie}
\label{sec:theorie}
\subsection{Das Bohr'sche Atommodell}
Das Atommodell nach Nils \textsc{Bohr} sagt voraus, dass Elektronen auf quantisierten Bahnen um den Atomkern kreisen.
Ohne die Quantenmechanik müsste durch die beschleunigte Ladung, da das Elektron auf der Kreisbahn der Zentripetalkraft ausgeliefert ist, elektromagnetische Strahlung ausgesand werden.
Dies würde bedeuten, dass das Elektron immer mehr Energie verliert und schließlich in den Atomkern fallen müsste.
Da dies offensichtlich nicht der Fall ist, entwickelte Bohr ausgehend von drei Postulaten sein Modell.
Die Bahnen, in den Elektronen strahlungsfrei kreisen können sind gequantelt.
Elektronen können von dem niedrigsten Energieniveau angeregt werden und in eine höhere Schale springen.
Die dafür benötigte Energie entspricht der Differenz der Energie der beiden Schalen.
Ein Lichtquant, welches diese besitzt, kann ein Elektron in genau diese Bahn anregen, indem es absorbiert wird.
Entspricht die Energie nicht einem Übergang, so können sie nicht aufgenommen werden.
Strahlt man nun durch einen Stoff, so sieht man charakteristische "`Einbrüche"' in der Intensität bei Energien, welche genau einer Anregung in eine höhere Schale entsprechen.
Relaxiert ein Elektron von einer höheren Bahn in eine energetisch niedrigere, so wird die Energie erneut in Form eines Lichtquants abgegeben, welcher die Energie hat, die es auch zur Anregung benötigte.

Die Energie eines Photons beträgt $E=h\nu$ mit der Planck-Konstanten $h=6.626\si{\joule\second}$ und der Frequenz $\nu$ des Photons.

Wird einem Elektron genügend Energie hinzugefügt, so kann es den Einflussbereich des Atomes verlassen.
Dieses ist nun positiv geladen und wird Ion genannt.

                                                                                                                                                                      
\section{Durchführung}
\label{sec:durchfuehrung}
In Abb. \ref{fig:aufbau} ist der Aufbau des Franck-Hertz-Versuches dargestellt.
Hierbei werden die Quecksilberatome nicht durch Photonen, sondern durch freie Elektronen angeregt.
Diese werden durch den glühelektrischen Effekt (s. Protokoll 12 - spezifische Elektronenladung) von einer Glühwendel emmitiert werden und dann mit einer elektrischen Spannung beschleunigt werden.
Zunächst werden sie 

\section{Auswertung}
\label{sec:auswertung}

\section{Diskussion}
\label{sec:diskussion}

\bibliography{literatur}
\bibliographystyle{babalpha}
\end{document}
