% Für Bindekorrektur als optionales Argument "BCORfaktormitmaßeinheit", dann
% sieht auch Option "twoside" vernünftig aus
% Näheres zu "scrartcl" bzw. "scrreprt" und "scrbook" siehe KOMA-Skript Doku
\documentclass[12pt,a4paper,titlepage,headinclude,bibtotoc]{scrartcl}


%---- Allgemeine Layout Einstellungen ------------------------------------------

% Für Kopf und Fußzeilen, siehe auch KOMA-Skript Doku
\usepackage[komastyle]{scrpage2}
\pagestyle{scrheadings}
\setheadsepline{0.5pt}[\color{black}]
\automark[section]{chapter}


%Einstellungen für Figuren- und Tabellenbeschriftungen
\setkomafont{captionlabel}{\sffamily\bfseries}
\setcapindent{0em}


%---- Weitere Pakete -----------------------------------------------------------
% Die Pakete sind alle in der TeX Live Distribution enthalten. Wichtige Adressen
% www.ctan.org, www.dante.de

% Sprachunterstützung
\usepackage[ngerman]{babel}

% Benutzung von Umlauten direkt im Text
% entweder "latin1" oder "utf8"
\usepackage[utf8]{inputenc}

% Pakete mit Mathesymbolen und zur Beseitigung von Schwächen der Mathe-Umgebung
\usepackage{latexsym,exscale,stmaryrd,amssymb,amsmath}

% Weitere Symbole
\usepackage[nointegrals]{wasysym}
\usepackage{eurosym}

% Anderes Literaturverzeichnisformat
%\usepackage[square,sort&compress]{natbib}

% Für Farbe
\usepackage{color}

% Zur Graphikausgabe
%Beipiel: \includegraphics[width=\textwidth]{grafik.png}
\usepackage{graphicx}

% Text umfließt Graphiken und Tabellen
% Beispiel:
% \begin{wrapfigure}[Zeilenanzahl]{"l" oder "r"}{breite}
%   \centering
%   \includegraphics[width=...]{grafik}
%   \caption{Beschriftung} 
%   \label{fig:grafik}
% \end{wrapfigure}
\usepackage{wrapfig}

% Mehrere Abbildungen nebeneinander
% Beispiel:
% \begin{figure}[htb]
%   \centering
%   \subfigure[Beschriftung 1\label{fig:label1}]
%   {\includegraphics[width=0.49\textwidth]{grafik1}}
%   \hfill
%   \subfigure[Beschriftung 2\label{fig:label2}]
%   {\includegraphics[width=0.49\textwidth]{grafik2}}
%   \caption{Beschriftung allgemein}
%   \label{fig:label-gesamt}
% \end{figure}
\usepackage{subfigure}

% Caption neben Abbildung
% Beispiel:
% \sidecaptionvpos{figure}{"c" oder "t" oder "b"}
% \begin{SCfigure}[rel. Breite (normalerweise = 1)][hbt]
%   \centering
%   \includegraphics[width=0.5\textwidth]{grafik.png}
%   \caption{Beschreibung}
%   \label{fig:}
% \end{SCfigure}
\usepackage{sidecap}

% Befehl für "Entspricht"-Zeichen
\newcommand{\corresponds}{\ensuremath{\mathrel{\widehat{=}}}}
% Befehl für Errorfunction
\newcommand{\erf}[1]{\text{ erf}\ensuremath{\left( #1 \right)}}

%Fußnoten zwingend auf diese Seite setzen
\interfootnotelinepenalty=1000

%Für chemische Formeln (von www.dante.de)
%% Anpassung an LaTeX(2e) von Bernd Raichle
\makeatletter
\DeclareRobustCommand{\chemical}[1]{%
  {\(\m@th
   \edef\resetfontdimens{\noexpand\)%
       \fontdimen16\textfont2=\the\fontdimen16\textfont2
       \fontdimen17\textfont2=\the\fontdimen17\textfont2\relax}%
   \fontdimen16\textfont2=2.7pt \fontdimen17\textfont2=2.7pt
   \mathrm{#1}%
   \resetfontdimens}}
\makeatother

%Honecker-Kasten mit $$\shadowbox{$xxxx$}$$
\usepackage{fancybox}

%SI-Package
\usepackage{siunitx}

%keine Einrückung, wenn Latex doppelte Leerzeile
\parindent0pt

%Bibliography \bibliography{literatur} und \cite{gerthsen}
%\usepackage{cite}
\usepackage{babelbib}
\selectbiblanguage{ngerman}

\begin{document}

\begin{titlepage}
\centering
\textsc{\Large Anfängerpraktikum der Fakultät für
  Physik,\\[1.5ex] Universität Göttingen}

\vspace*{3.2cm}

\rule{\textwidth}{1pt}\\[0.5cm]
{\huge \bfseries
  Versuch Nr. 22 Franck-Hertz-Versuch\\[1.5ex]
  Protokoll}\\[0.5cm]
\rule{\textwidth}{1pt}

\vspace*{2.5cm}

\begin{Large}
\begin{tabular}{ll}
Praktikant: &  Michael Lohmann\\
 &  Felix Kurtz\\
% &  Kevin Lüdemann\\
 E-Mail: & m.lohmann@stud.uni-goettingen.de\\
 &  felix.kurtz@stud.uni-goettingen.de\\
% &  kevin.luedemann@stud.uni-goettingen.de\\
 Betreuer: & Phillip Bastian\\
 Versuchsdatum: & 10.03.2015\\
\end{tabular}
\end{Large}

\vspace*{0.8cm}

\begin{Large}
\fbox{
  \begin{minipage}[t][2.5cm][t]{6cm} 
    Testat:
  \end{minipage}
}
\end{Large}

\end{titlepage}

\tableofcontents

\newpage

\section{Einleitung}
\label{sec:einleitung}
Das \textsc{Bohr}sche Atommodell war das erste, welches eine quantenmechanische Betrachtung des Atomes vornahm.
Viele Zeitgenossen sahen es daher eher kritisch.
Um so wichtiger, dass mit dem Franck-Hertz-Versuch 1913 erstmals eine experimentelle Überprüfung erfolgte.
Dies soll im folgenden mit einem etwas anderen Aufbau erneut geschehen.

\section{Theorie}
\label{sec:theorie}
\subsection{Das Bohr'sche Atommodell}
Das Atommodell nach Nils \textsc{Bohr} sagt voraus, dass Elektronen auf quantisierten Bahnen um den Atomkern kreisen.
Ohne die Quantenmechanik müsste durch die beschleunigte Ladung, da das Elektron auf der Kreisbahn der Zentripetalkraft ausgeliefert ist, elektromagnetische Strahlung ausgesand werden.
Dies würde bedeuten, dass das Elektron immer mehr Energie verliert und schließlich in den Atomkern fallen müsste.
Da dies offensichtlich nicht der Fall ist, entwickelte Bohr ausgehend von drei Postulaten sein Modell (nach \cite[S. 110 f.]{demtroeder3}):
\begin{itemize}
\item Der Drehimpulsbetrag des Elektrons ist quantisiert und kann nur in ganzzahligen vielfachen des Planckschen Wirkungsquantums $h$ vorkommen.
\item Die Bahnen des Elektrons um den Atomkern auf denen sie sich strahlungsfrei bewegen sind Kreise, deren Radien gequantelt sind.
\item Elektronen können durch Absorbtion von Photonen geeigneter Wellenlänge Energie aufnehmen und in energiereichere Bahnen gelangen.
Entspricht die Energie nicht mindestens einem Übergang, so können sie nicht aufgenommen werden ($\Delta E=h\nu$ mit der Planck-Konstanten $h=6.626\si{\joule\second}$).
Bei dem Relaxieren wird ein Photon eben dieser Wellenlänge emmitiert.
\end{itemize}

Statt mit Licht, kann man Atome auch mit Elektronen anregen.
Dies funktioniert genau so, nur dass die Energie des Elektrons durch die kinetische bestimmt ist.
Es können nun zwei Szenarien eintreten:
\begin{itemize}
\item Das Elektron hat weniger Energie, als für den kleinsten Übergang nötig wäre.
Es kann daher keine Energie abgeben und stößt so total elastisch.
Ledeglich seine Richtung kann sich geändert haben.
\item Hat das Elektron genügend Energie, so kann es bei einem Stoß ein Teil dessen abgeben um das Atom anzuregen.
Dies führt dazu, dass das Elektron Energie verliert und also langsamer wird.
Dies entspricht einem inelastischem Stoß.
\end{itemize}

Wird einem Elektron in einem Atom genügend Energie hinzugefügt, so kann es den Einflussbereich des Atomes verlassen.
Dieses ist nun positiv geladen und wird Ion genannt.
Die Energie, welche defür mindestens benötigt wird nennt man \textit{Ionisationsenergie}.

\subsection{Franck-Hertz-Versuch}
Strahlt man nun durch einen Stoff (hier Neon), so sieht man charakteristische "`Einbrüche"' in der Intensität bei Energien, welche genau einer Anregung in eine höhere Schale entsprechen.
Diese kann man vermessen um daraus Rückschlüsse über die Schalenstruktur der benutzten Atome zu erlangen.
Die erwartete Kurve der Elektronen-Energie und damit des Stromes ist in Abb. \ref{fig:theoKurve} zu sehen.
\begin{figure}[h]
\centering
\includegraphics{theoKurve}
\caption{Theoretisch erwareter Verlauf des Anodenstromes für eine Hg-Röhre von \cite[12.03.2015,16 Uhr]{lp22}}
\label{fig:theoKurve}
\end{figure}
Man tut dies, in dem man Elektronen mit einer Glühkathode emmitiert und zunächst von der Glühwendel mit einer Raumladungsspannung leicht beschleunigt.
Die hauptsächliche Beschleunigung findet jedoch erst zwischen den zwei Gittern durch die Beschleugigungsspannung $U_1$ statt.
Nachdem die Elektronen das zweite Gitter durchquert haben, werden sie von einer kleineren Bremsspannung erneut abgebremst.
Dies wird gemacht, damit nur Elektronen mit einer genügend hohen Energie die Anode erreichen können.

Die Elektronen haben nach einer Beschleunigung mit der Spannung $U$ eine Energie von 
\begin{align}
E=e\cdot U\label{eq:EU}
\end{align}
wobei nach \cite[S. 32]{demtroeder2} $e = \SI{1.6022e-19}{\coulomb}$ die Elementarladung darstellt.

Sind die Elektronen nun so stark beschleunigt, dass sie gerade die Atome anregen können, so hat die Wellenlänge des ausgesandten Lichts gerade die Frequenz
\begin{align}
e\cdot U=h\cdot \nu \label{eq:eUhnu}
\end{align}

Die nun verlangsamten Elektronen haben zunächst zu wenig Energie um die Bremsspannung zu überwinden.
Daher kommt zunächst kaum ein Stromfluss zustande (nur diejenigen Elektronen, welche keine Geschwindigkeit verloren haben kommen durch).
Ist die Strecke in der Röhre jedoch noch lang genug nach dem Stoß, so können die Elektronen wieder soweit beschleunigt werden, dass sie erneut die kritische Schwelle überschreiten können.


\section{Durchführung}
\label{sec:durchfuehrung}
\begin{figure}[h]
\centering
\includegraphics{RoehreNe}
\caption{Aufbau des Versuchs, modifiziert nach \cite[12.03.2015,16 Uhr]{lp22}}
\label{fig:aufbau}
\end{figure}
In Abb. \ref{fig:aufbau} ist der Aufbau des Franck-Hertz-Versuches dargestellt.
Hierbei werden die Neonatome nicht durch Photonen, sondern durch freie Elektronen angeregt.
Diese werden durch den glühelektrischen Effekt (s. Protokoll 12 - spezifische Elektronenladung) von einer Glühwendel emmitiert werden und dann mit einer elektrischen Spannung beschleunigt werden.
Zunächst werden Heizspannung $U_H$, Raumladungsspannung $U_3$ und Anodenbremsspannung $U_2$ wie auf dem Gerät vermerkt eingestellt.
Die Beschleunigungsspannung wird zu Beginn auf $0\si\volt$ gesetzt.
Obwohl in der Praktikumsanleitung keinerlei Hinweise auf ein Vorheizen des Neon-Aufbaus zu finden war, haben sich unsere Messwerte zu Beginn noch verändert, so dass dies sinnvoll zu sein scheint.

Um die Messung durchzuführen, ohne jedes mal zwischen den verschiedenen Ansichten des Steuerkastens hin- und herzuschalten, wird ein Voltmeter genutzt, um eine Ausgangsspannung zu vermessen, die proportional zum Anodenstrom ist.
Diese soll nun in Schritten $\Delta U_1=0.5\si\volt$ der Beschleunigungsspannung vermerkt werden.
Dies geschieht, bis $U_1=90\si\volt$ erreicht hat, oder eine Bogenentladung in der Röhre stattfindet.
Sollte dies geschehen, so schaltet sich das Steuergerät ab und die Messwerte nach erneutem Anschalten sind nicht mehr konsistent mit den vorher aufgenommenen.
Es ist daher davon abzusehen, testweise die Spannung zu erhöhen.

Bei der Durchführung kann man ab und zu einem Blick auf die Röhre werfen, da nach dem ersten Einbruch des Stromes dort eine schwach leuchtende Scheibe zu sehen ist.
Mit steigender Energie der Elektronen sind mehr Scheiben erkennbar, da diese wieder genügend beschleunigt werden um erneut Atome anzuregen.
Dies ist bei dem Aufbau mit der \chemical{Hg}-Röhre nicht der Fall, da dort die Emmision nicht im sichtbaren Bereich liegt.

\section{Auswertung}
\label{sec:auswertung}
\begin{figure}[h]
\centering
\input{messwerte}
\caption{Messwerte}
\label{fig:messwerte}
\end{figure}
In Abb. \ref{fig:messwerte} sind die Messwerte aufgetragen.
Die eingezeichneten Fehler betragen $(0.02+1\%)\si{\volt}$, da die Schwankungen in Bereichen von $0.02\si\volt$ liegen und das verwendete Messgerät eine Genauigkeit von $1\%$ besitzt.

\subsection{Maxima}
Die in Abb. \ref{fig:geraden} eingezeichneten Geraden sind je vom kleinsten bis zum größten Wert gefittet.
Die sich ergebenden Schnittpunkte sind in Tabelle \ref{tab:maxima} zu finden.
\begin{figure}[h]
\centering
\input{geraden}
\caption{Gefittete Geraden zur Bestimmung der Maxima}
\label{fig:geraden}
\end{figure}

\begin{table}
\centering
\begin{tabular}{|c|c|}
\hline Nr. des Maximums & Position [V] \\\hline\hline
1	& $17\pm 2 $ \\
2	& $34\pm 2 $ \\
3	& $51\pm 2 $ \\\hline
\end{tabular}
\caption{Position der Maxima berechnet aus den Schnittpunkten der Geraden aus Abb. \ref{fig:geraden}}
\label{tab:maxima}
\end{table}

Aus Tabelle \ref{tab:maxima} ergeben sich Differenzen von $\Delta U_{1,1}=(17\pm 3)\si\volt$ und $\Delta U_{1,2}=(17 \pm 3)\si\volt$ so dass sich insgesamt ein gewichteter Mittelwert von $\Delta U_1=(17\pm 2)\si\volt$ ergibt.
Daraus lässt sich nach Gleichung \eqref{eq:eUhnu} die emmitierte Wellenlänge berechnen:
\begin{align}
\frac{c}{\lambda}=\nu &= \frac{e\cdot\Delta U_1}{h}\notag\\
\Rightarrow \lambda&=\frac{c\cdot h}{e\cdot\Delta U_1}=(73\pm 9)\si{\nano\metre}
\end{align}
wobei der Fehler sich aus der Gauß'schen Fehlerfortpflanzung ergibt:
\begin{align}
\sigma_\lambda=\frac{c\cdot h}{e\cdot(\Delta U_1)^2}\sigma_{\Delta U_1}\quad .
\end{align}

Dies scheint auf den ersten Blick nicht mit der Tatsache übereinzustimmen, dass man rötliches Licht dort sehen kann.
Jedoch relaxiert nach \cite[13.03.2015, 19 Uhr]{UniWuppertal} das Elektron nicht direkt von $E\approx 18.9\si\electronvolt$ in den Grundzustand, sondern fällt zunächst in ein Energieniveau mit $E\approx 16.6\si{\electronvolt}$ und gibt dabei das sichtbare Licht ab.
Die letzte Emmission ist wieder im nicht sichtbaren Bereich.

\section{Diskussion}
\label{sec:diskussion}
Die Kurve aus Abb. \ref{fig:messwerte} zeigt die charakteristischen Merkmale der Franck-Hertz-Kurve.
Die Maxima in unserem Plot sind schlecht zu erkennen.
Dies kann mit einer höheren Bremsspannung optimiert werden, was jedoch wieder den Stromfluss verringert, wodurch die Fehler größer werden.

Bei der Messung fiel uns nach dem Wert $U_1=17.5\si\volt$ auf, dass der zuvor aufgenommene Wert der Messung von $U_1=15\si\volt$ um 4\% kleiner war, als bei der zweiten Aufnahme.
Daher haben wir ab $U_1=15\si\volt$ die Werte ein zweites Mal vermessen.
Dies kann man auch in der Kurve erkennen, da dort ein Knick zu sehen ist.
Die sich ändernden Messwerte lassen sich durch eine noch nicht vollständige Erwärmung der Röhre erklären, da spätere Kontrollen keinerlei nennenswerte Abweichung mehr feststellen konnten.
Dies spricht dafür, dass für erneute Messungen die Röhre ebenfalls vorgeheizt werden sollte, auch wenn dies laut Praktikumshandbuch bei dem Aufbau mit der Neon-Röhre nicht notwendig sein sollte.

Die berechnete emmitierte Wellenlänge hatte mit $(73\pm9)\si{\nano\metre}$ nur eine Abweichung $0.5$ bzw. $1.3\si{\nano\metre}$ von den Literaturwerten von $73.5$ und $74.3\si{\nano\metre}$.
Das das Fehlerintervall so groß ist, liegt an der Ungenauigkeit der Bestimmung der Maxima, da die Geradensteigungen Abweichungen von teilweise über $5\%$ hatten.
Dies hätte man durch eine weitere Einschränkung der Fit-Bereiche vermutlich noch verbessern können, da vor allem die ansteigenden Flanken eher weniger lineare Zusammenhänge zeigten.

Insgesammt werden die quantenmechanischen Eigenschaften von Neon jedoch sehr gut sichtbar.


\bibliography{literatur}
\bibliographystyle{babalpha}
\end{document}
